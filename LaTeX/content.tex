\begin{abstract}
 The abstract goes here...
\end{abstract}




%*****************************************
\chapter{Introduction}
%*****************************************
\hint{This chapter should motivate the thesis, provide a clear description of the problem to be solved, and describe the major contributions of this thesis. The chapter should have a length of about two pages!}

\section{Motivation}
What is the motivation for doing research in this area?

\section{Problem Statement and Contribution}
What is the problem that should be solved with this thesis?

\section{Outline}
How is the rest of this thesis structured?



%*****************************************
\chapter{Background}
\label{ch:background}
%*****************************************
\hint{This chapter should give a comprehensive overview on the background necessary to understand the thesis.
The chapter should have a length of about five pages!}


\section{Basics of Neural Networks\ \ \  \(WIP\)}
Neural Networks (NNs) are a part of most major AI-breakthrough in the last decade enabling computers to compete in fields formerly championed by humans.\footnote[1]{
	\begin{itemize}
		\item 
			2011: "Watson" of IBM defeats two former grand champions in "Jeopardy!" \cite{lally2011natural}
		\item 
			2011: "Siri" enables users to use natural language to interact with their phones 
			\cite{ARON201124}
		\item 
			2015: A convolutional neural network classifies images from the ImageNet dataset more accurately than human experts 
			\cite{Russakovsky2015} \cite{He_2015_ICCV}
		\item 
			2016: "AlphaGo" beats Lee Sedol, one of the world's strongest Go players
			\cite{gibney2016google} \cite{silver2017mastering}
	\end{itemize}
}
They implement a statistical understanding of AI, which is to say that they try to find a specific model optimizing the likelihood of reproducing input-output pairs similar to some training data. The competing philosophy directly divines behaviour rules, frequently from expert knowledge, and as such is far less dependant from data.  [citation needed]\\
For the former design model classes are the essential point of design. Two properties are sought after in the final model:
\begin{itemize}
	\item Richness: Ability to accurately separate different input-output "regions" in the training data
	\item Stability: Tendency to avoid sudden change in behaviour between and beyond given data points
\end{itemize}
Neural Networks form a model class that is both rich and easy to describe because they are comprised of many similar and simple units which combine to construct a large space of possible representations.[citation needed]\\
\begin{figure}
	\includegraphics[]{}
	\caption{Abstraction of a Neuron}
	\label{neuron1}
\end{figure}

\section{The Lottery Ticket Hypothesis}

\section{Basics of Natural Language Processing}

\section{Language Models \& Convolutional Neural Networks}


%*****************************************
\chapter{Related Work}
\label{ch:relatedwork}
%*****************************************
\hint{This chapter should give a comprehensive overview on the related work done by other authors followed by an analysis why the existing related work is not capable of solving the problem described in the introduction.
The chapter should have a length of about three to five pages!}
\section{Related Work Area 1}

\section{Related Work Area 2}

\section{Analysis of Related Work}

\section{Summary}

%*****************************************
\chapter{Design}
\label{ch:design}
%*****************************************
\hint{This chapter should describe the design of the own approach on a conceptional level without mentioning the implementation details. The section should have a length of about five pages.}

\section{Requirements and Assumptions}

\section{System Overview}

\subsection{Component 1}

\subsection{Component 2}

\section{Summary}

%*****************************************
\chapter{Implementation}
\label{ch:implementation}
%*****************************************

\hint{This chapter should describe the details of the implementation addressing the following questions: \\ \\
1. What are the design decisions made? \\
2. What is the environment the approach is developed in? \\
3. How are components mapped to classes of the source code? \\
4. How do the components interact with each other?  \\
5. What are limitations of the implementation? \\ \\
The section should have a length of about five pages.}
\section{Design Decisions}

\section{Architecture}

\section{Interaction of Components}

\section{Summary}

%*****************************************
\chapter{Evaluation}
\label{ch:evaluation}
%*****************************************
\hint{This chapter should describe how the evaluation of the implemented mechanism was done. \\ \\
1. Which evaluation method is used and why? Simulations, prototype? \\
2. What is the goal of the evaluation? Comparison? Proof of concept? \\
3. Wich metrics are used for characterizing the performance, costs, fairness, and efficiency of the system?\\
4. What are the parameter settings used in the evaluation and why? If possible always justify why a certain threshold has been chose for a particular parameter.  \\
5. What is the outcome of the evaluation? \\ \\
The section should have a length of about five to ten pages.}
\section{Goal and Methodology}

\section{Evaluation Setup}

\section{Evaluation Results}

\section{Analysis of Results}


%*****************************************
\chapter{Conclusions}
\label{ch:closure}
%*****************************************

\hint{This chapter should summarize the thesis and describe the main contributions of the thesis. Subsequently, it should describe possible future work in the context of the thesis. What are limitations of the developed solutions? Which things can be improved?
The section should have a length of about three pages.}

\section{Summary}

\section{Contributions}

\section{Future Work}

\section{Final Remarks}
